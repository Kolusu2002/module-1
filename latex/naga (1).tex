\documentclass[2pt,-letter paper]{article}
\usepackage[left=1in, right=0.75in, top=1in, bottom=0.75in]{geometry}
\usepackage{graphicx} % Required for inserting images
\usepackage{siunitx}
\usepackage{setspace}
\usepackage{gensymb}
\usepackage{xcolor}
\usepackage{caption}
%\usepackage{subcaption}
\doublespacing
\singlespacing
\usepackage[none]{hyphenat}
\usepackage{amssymb}
\usepackage{relsize}
\usepackage[cmex10]{amsmath}
\usepackage{mathtools}
\usepackage{amsmath}
\usepackage{commath}
\usepackage{amsthm}
\interdisplaylinepenalty=2500
%\savesymbol{iint}
\usepackage{txfonts}
%\restoresymbol{TXF}{iint}
\usepackage{wasysym}
\usepackage{amsthm}
\usepackage{mathrsfs}
\usepackage{txfonts}
\let\vec\mathbf{}
\usepackage{stfloats}
\usepackage{float}
\usepackage{cite}
\usepackage{cases}
\usepackage{subfig}
%\usepackage{xtab}
\usepackage{longtable}
\usepackage{multirow}
%\usepackage{algorithm}
\usepackage{amssymb}
%\usepackage{algpseudocode}
\usepackage{enumitem}
\usepackage{mathtools}
%\usepackage{eenrc}
%\usepackage[framemethod=tikz]{mdframed}
\usepackage{listings}
%\usepackage{listings}
\usepackage[latin1]{inputenc}
%%\usepackage{color}{   
%%\usepackage{lscape}
\usepackage{textcomp}
\usepackage{titling}
\usepackage{hyperref}
%\usepackage{fulbigskip}   
\usepackage{tikz}
\usepackage{graphicx}
\lstset{
  frame=single,
  breaklines=true
}
\let\vec\mathbf{}
\usepackage{enumitem}
\usepackage{graphicx}
\usepackage{siunitx}
\let\vec\mathbf{}
\usepackage{enumitem}
\usepackage{graphicx}
\usepackage{enumitem}
\usepackage{tfrupee}
\usepackage{amsmath}
\usepackage{amssymb}
\usepackage{mwe} % for blindtext and example-image-a in example
\usepackage{wrapfig}
\graphicspath{{figs/}}
\providecommand{\cbrak}[1]{\ensuremath{\left\{#1\right\}}}
\providecommand{\brak}[1]{\ensuremath{\left(#1\right)}}
\newcommand{\sgn}{\mathop{\mathrm{sgn}}}
\providecommand{\abs}[1]{\left\vert#1\right\vert}
\providecommand{\res}[1]{\Res\displaylimits_{#1}} 
\providecommand{\norm}[1]{\left\lVert#1\right\rVert}
%\providecommand{\norm}[1]{\lVert#1\rVert}
\providecommand{\mtx}[1]{\mathbf{#1}}
\providecommand{\mean}[1]{E\left[ #1 \right]}
\providecommand{\fourier}{\overset{\mathcal{F}}{ \rightleftharpoons}}
%\providecommand{\hilbert}{\overset{\mathcal{H}}{ \rightleftharpoons}}
\providecommand{\system}{\overset{\mathcal{H}}{ \longleftrightarrow}}
 %\newcommand{\solution}[2]{\textbf{Solution:}{#1}}
%\newcommand{\solution}{\noindent \textbf{Solution: }}
\newcommand{\cosec}{\,\text{cosec}\,}
\providecommand{\dec}[2]{\ensuremath{\overset{#1}{\underset{#2}{\gtrless}}}}
\newcommand{\myvec}[1]{\ensuremath{\begin{pmatrix}#1\end{pmatrix}}}
\newcommand{\myaugvec}[2]{\ensuremath{\begin{amatrix}{#1}#2\end{amatrix}}}
\newcommand{\mydet}[1]{\ensuremath{\begin{vmatrix}#1\end{vmatrix}}}
\title{MATHEMATICS}
\author{SECTION A}
\date{\today}
\begin{document}

\maketitle

\begin{enumerate}
\section{Matrix}
\item If $A$ is a square matrix satsifying $A'A=I$, write the value of $\vert A \vert$.
\item If $\vec{A}= \myvec{4&2\\-1& 1}$, show that$\brak{\vec{A}-2\vec{I}}\brak{\vec{A}-3\vec{I}}= 0$.
\item Show that for the matrix $\vec{A}=\myvec{1&1&1 \\ 1&2&-3 \\ 2&-1&3}$, ${A}^3-6{A}^2+5{A}+11{I}=0$. Hence, find $\vec{A^{-1}}$
	\item Using matrix method, solve the following system of equations :
		\begin{align*}
			{3x+2y+3z}=8 \\
			{2x+y-z}=1 \\
			{4x-3y+2z}=4 \\
		\end{align*}
\section{Differentiation}
\item Find the differential equation representing the family of curves $y=ae^{2x}+5$, where  is an arbitrary constant.
\item If $y=\cos(\sqrt{3})$, then find $\dfrac{dy}{dx}$.
\item Find the differential equation of the family of the curves $y = Ae^{2x} + Be^{-2x}$, where $A$ and $B$ are arbitrary constants.
\item If ${x} = ae^{t}\brak{\sin{t}+\cos{t}}$ and ${y} = ae^{t}\brak{\sin{t}-\cos{t}}$, then prove that $\dfrac{dy}{dx} = \dfrac{x+y}{x-y}$.
\item Differentiate $x^{\sin x} + \brak{\sin x}^{\cos x}$ with respect to $x$.
\section{Integration}
		\item Find : 
	\begin{align*}
		{\int {\frac{x-5}{\brak{x-3}^3}}e^{x}dx}
	\end{align*}
\item Find :
	\begin{align*}
		{\int{\frac{{\sin^{3}}{x} +{\cos^{3}}{x}}{{\sin^{2}}{x}{\cos^{2}}{x}}}dx} 
	\end{align*}
\item Find :                                       \begin{align*}                                           {\int {\frac{x-3}{\brak{x-1}^{3}}}e^{x}dx}   \end{align*}
\item Find :
\begin{align*}
	{\int{\frac{2\cos x}{\brak{1-\sin x}\brak{2-\cos^{2} x}}}dx}
\end{align*} 
\item Prove that
	\begin{align*}
	\int_{0}^{a} f\brak{x}dx=\int_{0}^{a} f\brak{a-x}dx
	\end{align*}
		and hence evalute
		\begin{align*}
		\int_{0}^{\frac{\pi}{2}} \frac{x}{{\sin x}+ {\cos x}}dx
		\end{align*}
\section{Vectors}
\item Find the direction cosines of a line which makes equal angles with the coordinate axes.
\item A line passes through the point with position vector $2\hat{i}-\hat{j}+4\hat{k}$ and is in the direction of the vector $\hat{i}+\hat{j}-2\hat{k}$. Find the equation of the line in cartesian form.
\item Show that the points $A(-2\hat{i}+3\hat{j}+5\hat{k})$, $B(\hat{i}+2\hat{j}+3\hat{k})$ and $C(7\hat{i}-\hat{k})$ are collinear.
\item Find $\mydet{\overrightarrow{a}\times \overrightarrow{b}}$, if $\overrightarrow{a}=2\hat{i}+\hat{j}+3\hat{k}$ and $\overrightarrow{b}=3\hat{i}+5\hat{j}-2\hat{k}$.
\item The scalar product of the vector $\overrightarrow{a}=\hat{i}+\hat{j}+\hat{k}$ with a unit vector along to sum of the vectors $\overrightarrow{b}=2\hat{i}+4\hat{j}-5\hat{k}$ and $\overrightarrow{c}=\lambda\hat{i}+2\hat{j}+3\hat{k}$ is equal to $1$. Find the value of $\lambda$ and hence find the unit vector along $\overrightarrow{b} + \overrightarrow{c}$.
\item Using method of integration, find the area of the triangle whose vertices are $\brak{1,0}, \brak{2,2}, \brak{3,1}$.
	\item Find the vector and cartesian equations of the plane passing through the points having position vectors $\hat{i} + \hat{j} - 2\hat{k}$, $2\hat{i} - \hat{j} + \hat{k}$ and $\hat{i} + 2\hat{j} + \hat{k}$. Write the equation of a plane passing through a point $\brak{2,3,7}$ and parallel to the plane obtained above. Hence, find the distance between the two parallel planes.

\section{Probability}
\item Find the probability distribution of $X$, the number of  heads in a simultaneous toss of two coins.
\item If $P(not A)=0.7$ and $P(B) = 0.7$ and $P(B\mid A) = 0.5$, then find $P(A\mid B)$.
\item A coin is tossed $5$ times. What is the probability of getting
	\begin{enumerate}
	\item $3$ heads.
	\item at most $3$ heads.
        \end{enumerate}
\item A bag contains $5$ red and $4$ black balls, a second bag contains $3$ red and $6$ black balls. One of the two bags is selected at random and two balls are drawn at random (without replacement) both of which are found to be red. Find the probability that the balls are drawn from the second bag.
\section{Functions}
\item Examine whether the opertion $\ast$ defined on $\mathbb{R}$, the set of all real numbers, by $a\ast b=\sqrt{a^2+b^2}$ is a binary operation or not, and if it is a binary operation, find whether it is associative or not.
\item Check whether the relation $R$ defined on the set $A=\cbrak{1,2,3,4,5,6}$ as $R =\cbrak {(a, b) : b = a + 1}$ is reflexive, symmetric or transitive.
\section{Intersection of Conics}
\item Find the equation of the normal to the curve $x^2 = 4y$ which passes through the point $\brak{1,4}$.
\section{Algebra}
\item Solve for $x$:
	\begin{align*}
	\tan^{-1}\brak{x+1}+\tan^{-1}\brak{x-1}=\tan^{-1}\brak{\dfrac{8}{31}}
	\end{align*}


\end{enumerate}
\end{document}
